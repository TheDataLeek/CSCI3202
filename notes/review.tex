\documentclass[12pt, landscape, twocolumn]{article}
%%%%%%%%%%%%%%%%%%%%%%%%%%%%%%%%
% FOR FUTURE REFERENCE ORDER FUCKING MATTERS GODDAMMIT
%%%%%%%%%%%%%%%%%%%%%%%%%%%%%%%%

%%%%%%%%%%%%%%%%%%%%%%%%%%%%%%%%%%%%%%%%%%%%%%%%%%%%%%%%%%%%%%%%%%%%%%%%%%%%%%%%
% LaTeX Imports
%%%%%%%%%%%%%%%%%%%%%%%%%%%%%%%%%%%%%%%%%%%%%%%%%%%%%%%%%%%%%%%%%%%%%%%%%%%%%%%%
\usepackage{amsfonts}                                                   % Math fonts
\usepackage{amsmath}                                                    % Math formatting
\usepackage{amssymb}                                                    % Math formatting
\usepackage{amsthm}                                                     % Math Theorems
%\usepackage{arydshln}                                                   % Dashed hlines
\usepackage{attachfile}                                                 % AttachFiles
\usepackage{cancel}                                                     % Cancelled math
\usepackage{caption}                                                    % Figure captioning
\usepackage{color}                                                      % Nice Colors
\usepackage[at]{easylist}                                        % Easy lists
\usepackage{fancyhdr}                                                   % Fancy Header
\usepackage[T1]{fontenc}                                                % Specific font-encoding
\usepackage[margin=1.5in, marginparwidth=2cm, marginparsep=2cm]{geometry} % Margins
\usepackage{graphicx}                                                   % Include images
\usepackage{hyperref}                                                   % Referencing
\usepackage[none]{hyphenat}                                             % Don't allow hyphenation
\usepackage{lipsum}                                                     % Lorem Ipsum Dummy Text
\usepackage{listings}                                                   % Code display
\usepackage{marginnote}                                                 % Notes in the margin
\usepackage{microtype}                                                  % Niceness
\usepackage{lib/minted}                                                 % Code display
\usepackage{./lib/mlptikz}                                              % Tikz mlp
\usepackage{multirow}                                                   % Multirow tables
\usepackage[framemethod=tikz]{mdframed}                                 % background color
\usepackage{pdfpages}                                                   % Include pdfs
\usepackage{pgfplots}                                                   % Create Pictures
\usepackage{rotating}                                                   % Figure rotation
\usepackage{setspace}                                                   % Allow double spacing
%\usepackage{subcaption}                                                 % Figure captioning
\usepackage{subfig}                                                 % Figure captioning
%\usepackage{tocloft}                                                    % List of Equations
\usepackage{longtable}                                                  % Huge Tables
\usepackage{./lib/multicol}                                             % Dynamic Mutlicolumns
\usepackage{supertabular}
\usepackage{float}
\setcounter{LTchunksize}{50}
%%%%%%%%%%%%%%%%%%%%%%%%%%%%%%%%%%%%%%%%%%%%%%%%%%%%%%%%%%%%%%%%%%%%%%%%%%%%%%%%
% Package Setup
%%%%%%%%%%%%%%%%%%%%%%%%%%%%%%%%%%%%%%%%%%%%%%%%%%%%%%%%%%%%%%%%%%%%%%%%%%%%%%%%
\hypersetup{%                                                           % Setup linking
    colorlinks=true,
    linkcolor=black,
    citecolor=black,
    filecolor=black,
    urlcolor=black,
}
\RequirePackage[l2tabu, orthodox]{nag}                                  % Nag about bad syntax
\renewcommand*\thesection{\arabic{section}}                             % Reset numbering
\renewcommand{\theFancyVerbLine}{{\arabic{FancyVerbLine}}}              % Needed for code display
\renewcommand{\footrulewidth}{0.4pt}                                    % Footer hline
\setcounter{secnumdepth}{3}                                             % Include subsubsections in numbering
\setcounter{tocdepth}{3}                                                % Include subsubsections in toc
%%%%%%%%%%%%%%%%%%%%%%%%%%%%%%%%%%%%%%%%%%%%%%%%%%%%%%%%%%%%%%%%%%%%%%%%%%%%%%%%
% Custom commands
%%%%%%%%%%%%%%%%%%%%%%%%%%%%%%%%%%%%%%%%%%%%%%%%%%%%%%%%%%%%%%%%%%%%%%%%%%%%%%%%
\newcommand{\nvec}[1]{\left\langle #1 \right\rangle}                    %  Easy to use vector
\newcommand{\inprod}[2]{\left\langle \vec{#1}, \vec{#2} \right\rangle}  %  Easy to use inner product
\newcommand{\norm}[1]{\lvert \lvert \vec{#1} \rvert \rvert}             %  Easy to use norm
\newcommand{\ma}[0]{\mathbf{A}}                                         %  Easy to use vector
\newcommand{\mb}[0]{\mathbf{B}}                                         %  Easy to use vector
\newcommand{\abs}[1]{\left\lvert #1 \right\rvert}                       %  Easy to use abs
\newcommand{\pren}[1]{\left( #1 \right)}                                %  Big parens
\newcommand{\Var}[0]{\text{Var}}                                %  Variance
\newcommand{\Cov}[0]{\text{Cov}}                                %  Variance
\newcommand{\Corr}[0]{\text{Corr}}                                %  Variance
\let\oldvec\vec
\renewcommand{\vec}[1]{\mathbf{#1}}                            %  Vector Styling
\newtheorem{thm}{Theorem}                                               %  Define the theorem name
\theoremstyle{definition}
\newtheorem{definition}{Definition}                                     %  Define the definition name
\newtheorem{ex}{Example}                                                %  Define the example name
\definecolor{bg}{rgb}{0.95,0.95,0.95}
\newminted{r}{frame=single,
              bgcolor=bg,
              linenos}
\newminted{python}{frame=single,
                   bgcolor=bg,
                   linenos}
\newcommand{\weave}{%
    \begin{centering}
    \begin{mdframed}[hidealllines=true,backgroundcolor=yellow!20]
    \hfill\begin{minipage}{\dimexpr\textwidth-1cm}}
\newcommand{\noweave}{%
    \xdef\tpd{\the\prevdepth}
    \end{minipage}\\
    \end{mdframed}
    \end{centering}}


\usepackage{genmpage}
\usepackage{pgfpages}

\geometry{margin=0.1in}

\pgfpagesphysicalpageoptions%
{
logical pages=6,
physical height=\paperwidth,
physical width=\paperheight,
}
\pgfpageslogicalpageoptions{1}
{
resized width=.4\pgfphysicalwidth,
resized height=0.45\pgfphysicalheight,
center=\pgfpoint{0.18\pgfphysicalwidth}{0.75\pgfphysicalheight}
}
\pgfpageslogicalpageoptions{2}
{
resized width=.4\pgfphysicalwidth,
resized height=0.45\pgfphysicalheight,
center=\pgfpoint{.18\pgfphysicalwidth}{.25\pgfphysicalheight}
}
\pgfpageslogicalpageoptions{3}
{
resized width=.4\pgfphysicalwidth,
resized height=0.45\pgfphysicalheight,
center=\pgfpoint{0.48\pgfphysicalwidth}{.75\pgfphysicalheight}
}
\pgfpageslogicalpageoptions{4}
{
resized width=.4\pgfphysicalwidth,
resized height=0.45\pgfphysicalheight,
center=\pgfpoint{.48\pgfphysicalwidth}{.25\pgfphysicalheight}
}
\pgfpageslogicalpageoptions{5}
{
resized width=.4\pgfphysicalwidth,
resized height=0.45\pgfphysicalheight,
center=\pgfpoint{.78\pgfphysicalwidth}{.75\pgfphysicalheight}
}
\pgfpageslogicalpageoptions{6}
{
resized width=.4\pgfphysicalwidth,
resized height=0.45\pgfphysicalheight,
center=\pgfpoint{.78\pgfphysicalwidth}{.25\pgfphysicalheight}
}

\pgfpageslogicalpageoptions{1}{border code=\pgfusepath{stroke}}
\pgfpageslogicalpageoptions{2}{border code=\pgfusepath{stroke}}
\pgfpageslogicalpageoptions{3}{border code=\pgfusepath{stroke}}
\pgfpageslogicalpageoptions{4}{border code=\pgfusepath{stroke}}
\pgfpageslogicalpageoptions{5}{border code=\pgfusepath{stroke}}
\pgfpageslogicalpageoptions{6}{border code=\pgfusepath{stroke}}
\pgfpageslogicalpageoptions{7}{border code=\pgfusepath{stroke}}
\pgfpageslogicalpageoptions{8}{border code=\pgfusepath{stroke}}
\pgfpageslogicalpageoptions{9}{border code=\pgfusepath{stroke}}
\pgfpageslogicalpageoptions{10}{border code=\pgfusepath{stroke}}
\pgfpageslogicalpageoptions{11}{border code=\pgfusepath{stroke}}
\pgfpageslogicalpageoptions{12}{border code=\pgfusepath{stroke}}





    
    
    \usepackage[T1]{fontenc}
    % Nicer default font than Computer Modern for most use cases
    \usepackage{palatino}

    % Basic figure setup, for now with no caption control since it's done
    % automatically by Pandoc (which extracts ![](path) syntax from Markdown).
    \usepackage{graphicx}
    % We will generate all images so they have a width \maxwidth. This means
    % that they will get their normal width if they fit onto the page, but
    % are scaled down if they would overflow the margins.
    \makeatletter
    \def\maxwidth{\ifdim\Gin@nat@width>\linewidth\linewidth
    \else\Gin@nat@width\fi}
    \makeatother
    \let\Oldincludegraphics\includegraphics
    % Set max figure width to be 80% of text width, for now hardcoded.
    \renewcommand{\includegraphics}[1]{\Oldincludegraphics[width=.8\maxwidth]{#1}}
    % Ensure that by default, figures have no caption (until we provide a
    % proper Figure object with a Caption API and a way to capture that
    % in the conversion process - todo).
    \usepackage{caption}
    \DeclareCaptionLabelFormat{nolabel}{}
    \captionsetup{labelformat=nolabel}

    \usepackage{adjustbox} % Used to constrain images to a maximum size 
    \usepackage{xcolor} % Allow colors to be defined
    \usepackage{enumerate} % Needed for markdown enumerations to work
    % Used to adjust the document margins
    \usepackage{amsmath} % Equations
    \usepackage{amssymb} % Equations
    \usepackage{textcomp} % defines textquotesingle
    % Hack from http://tex.stackexchange.com/a/47451/13684:
    \AtBeginDocument{%
        \def\PYZsq{\textquotesingle}% Upright quotes in Pygmentized code
    }
    \usepackage{upquote} % Upright quotes for verbatim code
    \usepackage{eurosym} % defines \euro
    \usepackage[mathletters]{ucs} % Extended unicode (utf-8) support
    \usepackage[utf8x]{inputenc} % Allow utf-8 characters in the tex document
    \usepackage{fancyvrb} % verbatim replacement that allows latex
    \usepackage{grffile} % extends the file name processing of package graphics 
                         % to support a larger range 
    % The hyperref package gives us a pdf with properly built
    % internal navigation ('pdf bookmarks' for the table of contents,
    % internal cross-reference links, web links for URLs, etc.)
    \usepackage{hyperref}
    \usepackage{longtable} % longtable support required by pandoc >1.10
    \usepackage{booktabs}  % table support for pandoc > 1.12.2
    \usepackage[normalem]{ulem} % ulem is needed to support strikethroughs (\sout)
                                % normalem makes italics be italics, not underlines
    

    
    
    % Colors for the hyperref package
    \definecolor{urlcolor}{rgb}{0,.145,.698}
    \definecolor{linkcolor}{rgb}{.71,0.21,0.01}
    \definecolor{citecolor}{rgb}{.12,.54,.11}

    % ANSI colors
    \definecolor{ansi-black}{HTML}{3E424D}
    \definecolor{ansi-black-intense}{HTML}{282C36}
    \definecolor{ansi-red}{HTML}{E75C58}
    \definecolor{ansi-red-intense}{HTML}{B22B31}
    \definecolor{ansi-green}{HTML}{00A250}
    \definecolor{ansi-green-intense}{HTML}{007427}
    \definecolor{ansi-yellow}{HTML}{DDB62B}
    \definecolor{ansi-yellow-intense}{HTML}{B27D12}
    \definecolor{ansi-blue}{HTML}{208FFB}
    \definecolor{ansi-blue-intense}{HTML}{0065CA}
    \definecolor{ansi-magenta}{HTML}{D160C4}
    \definecolor{ansi-magenta-intense}{HTML}{A03196}
    \definecolor{ansi-cyan}{HTML}{60C6C8}
    \definecolor{ansi-cyan-intense}{HTML}{258F8F}
    \definecolor{ansi-white}{HTML}{C5C1B4}
    \definecolor{ansi-white-intense}{HTML}{A1A6B2}

    % commands and environments needed by pandoc snippets
    % extracted from the output of `pandoc -s`
    \providecommand{\tightlist}{%
      \setlength{\itemsep}{0pt}\setlength{\parskip}{0pt}}
    \DefineVerbatimEnvironment{Highlighting}{Verbatim}{commandchars=\\\{\}}
    % Add ',fontsize=\small' for more characters per line
    \newenvironment{Shaded}{}{}
    \newcommand{\KeywordTok}[1]{\textcolor[rgb]{0.00,0.44,0.13}{\textbf{{#1}}}}
    \newcommand{\DataTypeTok}[1]{\textcolor[rgb]{0.56,0.13,0.00}{{#1}}}
    \newcommand{\DecValTok}[1]{\textcolor[rgb]{0.25,0.63,0.44}{{#1}}}
    \newcommand{\BaseNTok}[1]{\textcolor[rgb]{0.25,0.63,0.44}{{#1}}}
    \newcommand{\FloatTok}[1]{\textcolor[rgb]{0.25,0.63,0.44}{{#1}}}
    \newcommand{\CharTok}[1]{\textcolor[rgb]{0.25,0.44,0.63}{{#1}}}
    \newcommand{\StringTok}[1]{\textcolor[rgb]{0.25,0.44,0.63}{{#1}}}
    \newcommand{\CommentTok}[1]{\textcolor[rgb]{0.38,0.63,0.69}{\textit{{#1}}}}
    \newcommand{\OtherTok}[1]{\textcolor[rgb]{0.00,0.44,0.13}{{#1}}}
    \newcommand{\AlertTok}[1]{\textcolor[rgb]{1.00,0.00,0.00}{\textbf{{#1}}}}
    \newcommand{\FunctionTok}[1]{\textcolor[rgb]{0.02,0.16,0.49}{{#1}}}
    \newcommand{\RegionMarkerTok}[1]{{#1}}
    \newcommand{\ErrorTok}[1]{\textcolor[rgb]{1.00,0.00,0.00}{\textbf{{#1}}}}
    \newcommand{\NormalTok}[1]{{#1}}
    
    % Additional commands for more recent versions of Pandoc
    \newcommand{\ConstantTok}[1]{\textcolor[rgb]{0.53,0.00,0.00}{{#1}}}
    \newcommand{\SpecialCharTok}[1]{\textcolor[rgb]{0.25,0.44,0.63}{{#1}}}
    \newcommand{\VerbatimStringTok}[1]{\textcolor[rgb]{0.25,0.44,0.63}{{#1}}}
    \newcommand{\SpecialStringTok}[1]{\textcolor[rgb]{0.73,0.40,0.53}{{#1}}}
    \newcommand{\ImportTok}[1]{{#1}}
    \newcommand{\DocumentationTok}[1]{\textcolor[rgb]{0.73,0.13,0.13}{\textit{{#1}}}}
    \newcommand{\AnnotationTok}[1]{\textcolor[rgb]{0.38,0.63,0.69}{\textbf{\textit{{#1}}}}}
    \newcommand{\CommentVarTok}[1]{\textcolor[rgb]{0.38,0.63,0.69}{\textbf{\textit{{#1}}}}}
    \newcommand{\VariableTok}[1]{\textcolor[rgb]{0.10,0.09,0.49}{{#1}}}
    \newcommand{\ControlFlowTok}[1]{\textcolor[rgb]{0.00,0.44,0.13}{\textbf{{#1}}}}
    \newcommand{\OperatorTok}[1]{\textcolor[rgb]{0.40,0.40,0.40}{{#1}}}
    \newcommand{\BuiltInTok}[1]{{#1}}
    \newcommand{\ExtensionTok}[1]{{#1}}
    \newcommand{\PreprocessorTok}[1]{\textcolor[rgb]{0.74,0.48,0.00}{{#1}}}
    \newcommand{\AttributeTok}[1]{\textcolor[rgb]{0.49,0.56,0.16}{{#1}}}
    \newcommand{\InformationTok}[1]{\textcolor[rgb]{0.38,0.63,0.69}{\textbf{\textit{{#1}}}}}
    \newcommand{\WarningTok}[1]{\textcolor[rgb]{0.38,0.63,0.69}{\textbf{\textit{{#1}}}}}
    
    
    % Define a nice break command that doesn't care if a line doesn't already
    % exist.
    \def\br{\hspace*{\fill} \\* }
    % Math Jax compatability definitions
    \def\gt{>}
    \def\lt{<}
    % Document parameters
    \title{Notes}
    
    
    

    % Pygments definitions
    
\makeatletter
\def\PY@reset{\let\PY@it=\relax \let\PY@bf=\relax%
    \let\PY@ul=\relax \let\PY@tc=\relax%
    \let\PY@bc=\relax \let\PY@ff=\relax}
\def\PY@tok#1{\csname PY@tok@#1\endcsname}
\def\PY@toks#1+{\ifx\relax#1\empty\else%
    \PY@tok{#1}\expandafter\PY@toks\fi}
\def\PY@do#1{\PY@bc{\PY@tc{\PY@ul{%
    \PY@it{\PY@bf{\PY@ff{#1}}}}}}}
\def\PY#1#2{\PY@reset\PY@toks#1+\relax+\PY@do{#2}}

\expandafter\def\csname PY@tok@gd\endcsname{\def\PY@tc##1{\textcolor[rgb]{0.63,0.00,0.00}{##1}}}
\expandafter\def\csname PY@tok@gu\endcsname{\let\PY@bf=\textbf\def\PY@tc##1{\textcolor[rgb]{0.50,0.00,0.50}{##1}}}
\expandafter\def\csname PY@tok@gt\endcsname{\def\PY@tc##1{\textcolor[rgb]{0.00,0.27,0.87}{##1}}}
\expandafter\def\csname PY@tok@gs\endcsname{\let\PY@bf=\textbf}
\expandafter\def\csname PY@tok@gr\endcsname{\def\PY@tc##1{\textcolor[rgb]{1.00,0.00,0.00}{##1}}}
\expandafter\def\csname PY@tok@cm\endcsname{\let\PY@it=\textit\def\PY@tc##1{\textcolor[rgb]{0.25,0.50,0.50}{##1}}}
\expandafter\def\csname PY@tok@vg\endcsname{\def\PY@tc##1{\textcolor[rgb]{0.10,0.09,0.49}{##1}}}
\expandafter\def\csname PY@tok@vi\endcsname{\def\PY@tc##1{\textcolor[rgb]{0.10,0.09,0.49}{##1}}}
\expandafter\def\csname PY@tok@mh\endcsname{\def\PY@tc##1{\textcolor[rgb]{0.40,0.40,0.40}{##1}}}
\expandafter\def\csname PY@tok@cs\endcsname{\let\PY@it=\textit\def\PY@tc##1{\textcolor[rgb]{0.25,0.50,0.50}{##1}}}
\expandafter\def\csname PY@tok@ge\endcsname{\let\PY@it=\textit}
\expandafter\def\csname PY@tok@vc\endcsname{\def\PY@tc##1{\textcolor[rgb]{0.10,0.09,0.49}{##1}}}
\expandafter\def\csname PY@tok@il\endcsname{\def\PY@tc##1{\textcolor[rgb]{0.40,0.40,0.40}{##1}}}
\expandafter\def\csname PY@tok@go\endcsname{\def\PY@tc##1{\textcolor[rgb]{0.53,0.53,0.53}{##1}}}
\expandafter\def\csname PY@tok@cp\endcsname{\def\PY@tc##1{\textcolor[rgb]{0.74,0.48,0.00}{##1}}}
\expandafter\def\csname PY@tok@gi\endcsname{\def\PY@tc##1{\textcolor[rgb]{0.00,0.63,0.00}{##1}}}
\expandafter\def\csname PY@tok@gh\endcsname{\let\PY@bf=\textbf\def\PY@tc##1{\textcolor[rgb]{0.00,0.00,0.50}{##1}}}
\expandafter\def\csname PY@tok@ni\endcsname{\let\PY@bf=\textbf\def\PY@tc##1{\textcolor[rgb]{0.60,0.60,0.60}{##1}}}
\expandafter\def\csname PY@tok@nl\endcsname{\def\PY@tc##1{\textcolor[rgb]{0.63,0.63,0.00}{##1}}}
\expandafter\def\csname PY@tok@nn\endcsname{\let\PY@bf=\textbf\def\PY@tc##1{\textcolor[rgb]{0.00,0.00,1.00}{##1}}}
\expandafter\def\csname PY@tok@no\endcsname{\def\PY@tc##1{\textcolor[rgb]{0.53,0.00,0.00}{##1}}}
\expandafter\def\csname PY@tok@na\endcsname{\def\PY@tc##1{\textcolor[rgb]{0.49,0.56,0.16}{##1}}}
\expandafter\def\csname PY@tok@nb\endcsname{\def\PY@tc##1{\textcolor[rgb]{0.00,0.50,0.00}{##1}}}
\expandafter\def\csname PY@tok@nc\endcsname{\let\PY@bf=\textbf\def\PY@tc##1{\textcolor[rgb]{0.00,0.00,1.00}{##1}}}
\expandafter\def\csname PY@tok@nd\endcsname{\def\PY@tc##1{\textcolor[rgb]{0.67,0.13,1.00}{##1}}}
\expandafter\def\csname PY@tok@ne\endcsname{\let\PY@bf=\textbf\def\PY@tc##1{\textcolor[rgb]{0.82,0.25,0.23}{##1}}}
\expandafter\def\csname PY@tok@nf\endcsname{\def\PY@tc##1{\textcolor[rgb]{0.00,0.00,1.00}{##1}}}
\expandafter\def\csname PY@tok@si\endcsname{\let\PY@bf=\textbf\def\PY@tc##1{\textcolor[rgb]{0.73,0.40,0.53}{##1}}}
\expandafter\def\csname PY@tok@s2\endcsname{\def\PY@tc##1{\textcolor[rgb]{0.73,0.13,0.13}{##1}}}
\expandafter\def\csname PY@tok@nt\endcsname{\let\PY@bf=\textbf\def\PY@tc##1{\textcolor[rgb]{0.00,0.50,0.00}{##1}}}
\expandafter\def\csname PY@tok@nv\endcsname{\def\PY@tc##1{\textcolor[rgb]{0.10,0.09,0.49}{##1}}}
\expandafter\def\csname PY@tok@s1\endcsname{\def\PY@tc##1{\textcolor[rgb]{0.73,0.13,0.13}{##1}}}
\expandafter\def\csname PY@tok@ch\endcsname{\let\PY@it=\textit\def\PY@tc##1{\textcolor[rgb]{0.25,0.50,0.50}{##1}}}
\expandafter\def\csname PY@tok@m\endcsname{\def\PY@tc##1{\textcolor[rgb]{0.40,0.40,0.40}{##1}}}
\expandafter\def\csname PY@tok@gp\endcsname{\let\PY@bf=\textbf\def\PY@tc##1{\textcolor[rgb]{0.00,0.00,0.50}{##1}}}
\expandafter\def\csname PY@tok@sh\endcsname{\def\PY@tc##1{\textcolor[rgb]{0.73,0.13,0.13}{##1}}}
\expandafter\def\csname PY@tok@ow\endcsname{\let\PY@bf=\textbf\def\PY@tc##1{\textcolor[rgb]{0.67,0.13,1.00}{##1}}}
\expandafter\def\csname PY@tok@sx\endcsname{\def\PY@tc##1{\textcolor[rgb]{0.00,0.50,0.00}{##1}}}
\expandafter\def\csname PY@tok@bp\endcsname{\def\PY@tc##1{\textcolor[rgb]{0.00,0.50,0.00}{##1}}}
\expandafter\def\csname PY@tok@c1\endcsname{\let\PY@it=\textit\def\PY@tc##1{\textcolor[rgb]{0.25,0.50,0.50}{##1}}}
\expandafter\def\csname PY@tok@o\endcsname{\def\PY@tc##1{\textcolor[rgb]{0.40,0.40,0.40}{##1}}}
\expandafter\def\csname PY@tok@kc\endcsname{\let\PY@bf=\textbf\def\PY@tc##1{\textcolor[rgb]{0.00,0.50,0.00}{##1}}}
\expandafter\def\csname PY@tok@c\endcsname{\let\PY@it=\textit\def\PY@tc##1{\textcolor[rgb]{0.25,0.50,0.50}{##1}}}
\expandafter\def\csname PY@tok@mf\endcsname{\def\PY@tc##1{\textcolor[rgb]{0.40,0.40,0.40}{##1}}}
\expandafter\def\csname PY@tok@err\endcsname{\def\PY@bc##1{\setlength{\fboxsep}{0pt}\fcolorbox[rgb]{1.00,0.00,0.00}{1,1,1}{\strut ##1}}}
\expandafter\def\csname PY@tok@mb\endcsname{\def\PY@tc##1{\textcolor[rgb]{0.40,0.40,0.40}{##1}}}
\expandafter\def\csname PY@tok@ss\endcsname{\def\PY@tc##1{\textcolor[rgb]{0.10,0.09,0.49}{##1}}}
\expandafter\def\csname PY@tok@sr\endcsname{\def\PY@tc##1{\textcolor[rgb]{0.73,0.40,0.53}{##1}}}
\expandafter\def\csname PY@tok@mo\endcsname{\def\PY@tc##1{\textcolor[rgb]{0.40,0.40,0.40}{##1}}}
\expandafter\def\csname PY@tok@kd\endcsname{\let\PY@bf=\textbf\def\PY@tc##1{\textcolor[rgb]{0.00,0.50,0.00}{##1}}}
\expandafter\def\csname PY@tok@mi\endcsname{\def\PY@tc##1{\textcolor[rgb]{0.40,0.40,0.40}{##1}}}
\expandafter\def\csname PY@tok@kn\endcsname{\let\PY@bf=\textbf\def\PY@tc##1{\textcolor[rgb]{0.00,0.50,0.00}{##1}}}
\expandafter\def\csname PY@tok@cpf\endcsname{\let\PY@it=\textit\def\PY@tc##1{\textcolor[rgb]{0.25,0.50,0.50}{##1}}}
\expandafter\def\csname PY@tok@kr\endcsname{\let\PY@bf=\textbf\def\PY@tc##1{\textcolor[rgb]{0.00,0.50,0.00}{##1}}}
\expandafter\def\csname PY@tok@s\endcsname{\def\PY@tc##1{\textcolor[rgb]{0.73,0.13,0.13}{##1}}}
\expandafter\def\csname PY@tok@kp\endcsname{\def\PY@tc##1{\textcolor[rgb]{0.00,0.50,0.00}{##1}}}
\expandafter\def\csname PY@tok@w\endcsname{\def\PY@tc##1{\textcolor[rgb]{0.73,0.73,0.73}{##1}}}
\expandafter\def\csname PY@tok@kt\endcsname{\def\PY@tc##1{\textcolor[rgb]{0.69,0.00,0.25}{##1}}}
\expandafter\def\csname PY@tok@sc\endcsname{\def\PY@tc##1{\textcolor[rgb]{0.73,0.13,0.13}{##1}}}
\expandafter\def\csname PY@tok@sb\endcsname{\def\PY@tc##1{\textcolor[rgb]{0.73,0.13,0.13}{##1}}}
\expandafter\def\csname PY@tok@k\endcsname{\let\PY@bf=\textbf\def\PY@tc##1{\textcolor[rgb]{0.00,0.50,0.00}{##1}}}
\expandafter\def\csname PY@tok@se\endcsname{\let\PY@bf=\textbf\def\PY@tc##1{\textcolor[rgb]{0.73,0.40,0.13}{##1}}}
\expandafter\def\csname PY@tok@sd\endcsname{\let\PY@it=\textit\def\PY@tc##1{\textcolor[rgb]{0.73,0.13,0.13}{##1}}}

\def\PYZbs{\char`\\}
\def\PYZus{\char`\_}
\def\PYZob{\char`\{}
\def\PYZcb{\char`\}}
\def\PYZca{\char`\^}
\def\PYZam{\char`\&}
\def\PYZlt{\char`\<}
\def\PYZgt{\char`\>}
\def\PYZsh{\char`\#}
\def\PYZpc{\char`\%}
\def\PYZdl{\char`\$}
\def\PYZhy{\char`\-}
\def\PYZsq{\char`\'}
\def\PYZdq{\char`\"}
\def\PYZti{\char`\~}
% for compatibility with earlier versions
\def\PYZat{@}
\def\PYZlb{[}
\def\PYZrb{]}
\makeatother


    % Exact colors from NB
    \definecolor{incolor}{rgb}{0.0, 0.0, 0.5}
    \definecolor{outcolor}{rgb}{0.545, 0.0, 0.0}



    
    % Prevent overflowing lines due to hard-to-break entities
    \sloppy 
    % Setup hyperref package
    \hypersetup{
      breaklinks=true,  % so long urls are correctly broken across lines
      colorlinks=true,
      urlcolor=urlcolor,
      linkcolor=linkcolor,
      citecolor=citecolor,
      }
    % Slightly bigger margins than the latex defaults
    
    \geometry{verbose,tmargin=1in,bmargin=1in,lmargin=1in,rmargin=1in}
    
    

    \begin{document}
    
    
    \maketitle
    
    

    
    \section{Search Algorithms}\label{search-algorithms}

    \subsection{Breadth First Search (BFS)}\label{breadth-first-search-bfs}

Here's the
\href{https://en.wikipedia.org/wiki/Breadth-first_search}{Wikipedia}
link.

TL;DR Get further and further out as you search.

\(O(|V| + |E|)\)

\begin{Shaded}
\begin{Highlighting}[]
\KeywordTok{def} \NormalTok{BFS(graph, root):}
  \ControlFlowTok{for} \NormalTok{node }\KeywordTok{in} \NormalTok{graph:}
    \NormalTok{node.distance }\OperatorTok{=} \NormalTok{inf}
    \NormalTok{node.parent }\OperatorTok{=} \VariableTok{None}
  \NormalTok{q }\OperatorTok{=} \NormalTok{Queue()}
  \NormalTok{root.distance }\OperatorTok{=} \DecValTok{0}
  \NormalTok{q.put(root)}
  \ControlFlowTok{while} \KeywordTok{not} \NormalTok{q.empty():}
    \NormalTok{cnode }\OperatorTok{=} \NormalTok{q.get()}
    \ControlFlowTok{for} \NormalTok{node }\KeywordTok{in} \NormalTok{cnode.adjacent:}
      \ControlFlowTok{if} \NormalTok{node.distance }\OperatorTok{=} \NormalTok{inf:}
        \NormalTok{node.distance }\OperatorTok{=} \NormalTok{cnode.distance }\OperatorTok{+} \DecValTok{1}
        \NormalTok{node.parent }\OperatorTok{=} \NormalTok{cnode}
        \NormalTok{q.put(node)}
\end{Highlighting}
\end{Shaded}

    \subsection{Depth First Search (DFS)}\label{depth-first-search-dfs}

\href{https://en.wikipedia.org/wiki/Depth-first_search}{Wikipedia}.
TL;DR search all the way down before going out.

\begin{Shaded}
\begin{Highlighting}[]
\KeywordTok{def} \NormalTok{DFS(root):}
  \NormalTok{root.discovered }\OperatorTok{=} \VariableTok{True}
  \ControlFlowTok{for} \NormalTok{node }\KeywordTok{in} \NormalTok{root.adjacent:}
    \ControlFlowTok{if} \NormalTok{node.discoverd }\OperatorTok{==} \VariableTok{False}\NormalTok{:}
      \NormalTok{DFS(node)}
\end{Highlighting}
\end{Shaded}

    \section{Heuristics}\label{heuristics}

\href{https://en.wikipedia.org/wiki/Heuristic_(computer_science)}{Wikipedia}.
Rank algorithm based on information available.

    \section{Graph Traversal}\label{graph-traversal}

    \subsection{Dijkstra's Algorithm}\label{dijkstras-algorithm}

\href{https://en.wikipedia.org/wiki/Dijkstra\%27s_algorithm}{Wikipedia}

\begin{Shaded}
\begin{Highlighting}[]
\KeywordTok{def} \NormalTok{dijkstra(graph, start_node, end_node):}
  \ControlFlowTok{for} \NormalTok{node }\KeywordTok{in} \NormalTok{graph:}
    \NormalTok{node.distance }\OperatorTok{=} \NormalTok{inf}
    \NormalTok{node.visited }\OperatorTok{=} \VariableTok{False}
  \NormalTok{start_node.distance }\OperatorTok{=} \DecValTok{0}
  \NormalTok{cnode }\OperatorTok{=} \NormalTok{start_node}
  \CommentTok{# Using min-heap as unvisited helps}
  \CommentTok{# with runtime, cause it's always sorted}
  \NormalTok{unvisited }\OperatorTok{=} \NormalTok{\{n }\ControlFlowTok{for} \NormalTok{n }\KeywordTok{in} \NormalTok{graph.nodes }\ControlFlowTok{if} \NormalTok{n.visited }\OperatorTok{==} \VariableTok{False}\NormalTok{\}}
  \ControlFlowTok{while} \VariableTok{True}\NormalTok{:}
    \ControlFlowTok{for} \NormalTok{node }\KeywordTok{in} \NormalTok{cnode.adjacent:}
      \NormalTok{tentative_distance }\OperatorTok{=} \NormalTok{cnode.distance }\OperatorTok{+} \DecValTok{1} \CommentTok{# or edge weight}
      \ControlFlowTok{if} \NormalTok{tentative_distance }\OperatorTok{<} \NormalTok{node.distance:}
        \NormalTok{node.distance }\OperatorTok{=} \NormalTok{tentative_distance}
      \NormalTok{cnode.visited }\OperatorTok{=} \VariableTok{True}
      \NormalTok{unvisited.remove(cnode)}
      \ControlFlowTok{if} \NormalTok{cnode }\OperatorTok{==} \NormalTok{end_node:}
        \ControlFlowTok{break}
      \NormalTok{cnode }\OperatorTok{=} \NormalTok{unvisited.smallest_distance}
\end{Highlighting}
\end{Shaded}

    \subsection{A* Search}\label{a-search}

\href{https://en.wikipedia.org/wiki/A*_search_algorithm}{Wikipedia} -
very similar to Dijkstra\ldots{}

\begin{Shaded}
\begin{Highlighting}[]
\KeywordTok{def} \NormalTok{A_star(graph, start, end):}
  \NormalTok{closed_set }\OperatorTok{=} \NormalTok{\{\}  }\CommentTok{# Nodes already evaluated}
  \NormalTok{open_set }\OperatorTok{=} \NormalTok{\{\}  }\CommentTok{# Nodes discovered to be evaluated}
  \NormalTok{came_from }\OperatorTok{=} \VariableTok{None}  \CommentTok{# for each node, its most efficient parent}
  \ControlFlowTok{for} \NormalTok{node }\KeywordTok{in} \NormalTok{graph:}
    \NormalTok{node.gscore }\OperatorTok{=} \NormalTok{inf  }\CommentTok{# cost of getting from start to this node}
    \NormalTok{node.fscore }\OperatorTok{=} \NormalTok{inf  }\CommentTok{# cost of start to goal through node}
  \NormalTok{start.gscore }\OperatorTok{=} \DecValTok{0}
  \CommentTok{# start's fscore is completely heuristic}
  \NormalTok{start.fscore }\OperatorTok{=} \NormalTok{heuristic_estimate(start, goal)}
  \ControlFlowTok{while} \BuiltInTok{len}\NormalTok{(open_set) }\OperatorTok{!=} \DecValTok{0}\NormalTok{:}
    \NormalTok{cnode }\OperatorTok{=} \NormalTok{lowest_fscore(open_set)}
    \ControlFlowTok{if} \NormalTok{cnode }\OperatorTok{==} \NormalTok{end:}
      \ControlFlowTok{return} \NormalTok{reconstruct_path(camefrom, cnode)}
    \NormalTok{open_set.remove(cnode)}
    \NormalTok{closed_set.add(cnode)}
    \ControlFlowTok{for} \NormalTok{node }\KeywordTok{in} \NormalTok{cnode.adjacent:}
      \ControlFlowTok{if} \NormalTok{node }\KeywordTok{in} \NormalTok{closed_set:}
        \BuiltInTok{next} \CommentTok{# ignore evaluated neighbors}
      \NormalTok{tmp_gscore }\OperatorTok{=} \NormalTok{cnode.gscore }\OperatorTok{+} \NormalTok{dist_between(cnode, node)}
      \ControlFlowTok{if} \NormalTok{node }\KeywordTok{not} \KeywordTok{in} \NormalTok{open_set:}
        \NormalTok{open_set.add(node) }\CommentTok{# discover a new node}
      \ControlFlowTok{elif} \NormalTok{tmp_gscore }\OperatorTok{>=} \NormalTok{node.gscore:}
        \BuiltInTok{next} \CommentTok{# this is not a better path}
      \CommentTok{# Best path until now}
      \NormalTok{cameFrom[node] }\OperatorTok{=} \NormalTok{cnode}
      \NormalTok{node.gscore }\OperatorTok{=} \NormalTok{tmp_gscore}
      \NormalTok{node.fscore }\OperatorTok{=} \NormalTok{node.gscore }\OperatorTok{+} \NormalTok{heuristic_estimate(node, end)}
  \ControlFlowTok{return} \VariableTok{False}
  
\KeywordTok{def} \NormalTok{reconstruct(camefrom, current):}
  \CommentTok{# basically go backwards from end to start}
  \NormalTok{total_path }\OperatorTok{=} \NormalTok{[current]}
  \ControlFlowTok{while} \NormalTok{current }\KeywordTok{in} \NormalTok{cameFrom.keys:}
    \NormalTok{current }\OperatorTok{=} \NormalTok{camefrom[current]}
    \NormalTok{total_path.append(current)}
  \ControlFlowTok{return} \NormalTok{total_path}
\end{Highlighting}
\end{Shaded}

    \section{Adversarial Search}\label{adversarial-search}

    \subsection{Minimax}\label{minimax}

\begin{verbatim}
minimax(node, depth, maxPlayer)
    if depth == 0 or terminal(node) //terminal test is true
        return f(node)  //evaluation of the node
    if maxPlayer //Player(s) = MAX
        bestValue = -MAX_INT //system property, maximum negative integer
        for each child in node.adjacent
            eval = minimax(child, depth - 1, FALSE)
            print eval
            bestValue = max(bestValue, eval)
        return bestValue
    else //Player(s) = MIN
        bestValue = MAX_INT
        for each child in node.adjacent
            eval = minimax(child, depth - 1, TRUE)
            print eval
            bestValue = min(bestValue, eval)
        return bestValue

minimax(origin, depth, TRUE) //call from root for MAX player
\end{verbatim}

    \subsection{\texorpdfstring{Minimax with \(\alpha\)-\(\beta\)
Pruning}{Minimax with \textbackslash{}alpha-\textbackslash{}beta Pruning}}\label{minimax-with-alpha-beta-pruning}

\begin{Shaded}
\begin{Highlighting}[]
\KeywordTok{def} \NormalTok{alpha_beta_pruning(root, depth, player, alpha, beta):}
    \ControlFlowTok{if} \NormalTok{depth }\OperatorTok{==} \DecValTok{0} \KeywordTok{or} \NormalTok{root.is_empty:}
        \ControlFlowTok{return} \NormalTok{root.value}
    \ControlFlowTok{if} \NormalTok{player }\OperatorTok{==} \StringTok{'MAX'}\NormalTok{:}
        \NormalTok{best }\OperatorTok{=} \OperatorTok{-}\NormalTok{inf}
        \NormalTok{pruned }\OperatorTok{=} \VariableTok{False}
        \ControlFlowTok{for} \NormalTok{child }\KeywordTok{in} \NormalTok{root.children:}
            \ControlFlowTok{if} \NormalTok{pruned:}
                \NormalTok{child.pruned }\OperatorTok{=} \VariableTok{True}
            \ControlFlowTok{else}\NormalTok{:}
                \NormalTok{best }\OperatorTok{=} \BuiltInTok{max}\NormalTok{(best,}
                           \NormalTok{alpha_beta_pruning(child, depth }\OperatorTok{-} \DecValTok{1}\NormalTok{, }\StringTok{'MIN'}\NormalTok{,}
                                              \NormalTok{alpha, beta))}
                \NormalTok{alpha }\OperatorTok{=} \BuiltInTok{max}\NormalTok{(alpha, best)}
                \NormalTok{root.alpha }\OperatorTok{=} \NormalTok{alpha}
                \ControlFlowTok{if} \NormalTok{beta }\OperatorTok{<=} \NormalTok{alpha:}
                    \NormalTok{pruned }\OperatorTok{=} \VariableTok{True}
        \BuiltInTok{print}\NormalTok{(root)}
        \ControlFlowTok{return} \NormalTok{best}
    \ControlFlowTok{else}\NormalTok{:}
        \NormalTok{best }\OperatorTok{=} \NormalTok{inf}
        \NormalTok{pruned }\OperatorTok{=} \VariableTok{False}
        \ControlFlowTok{for} \NormalTok{child }\KeywordTok{in} \NormalTok{root.children:}
            \ControlFlowTok{if} \NormalTok{pruned:}
                \NormalTok{child.pruned }\OperatorTok{=} \VariableTok{True}
            \ControlFlowTok{else}\NormalTok{:}
                \NormalTok{best }\OperatorTok{=} \BuiltInTok{min}\NormalTok{(best,}
                           \NormalTok{alpha_beta_pruning(child, depth }\OperatorTok{-} \DecValTok{1}\NormalTok{, }\StringTok{'MAX'}\NormalTok{,}
                                              \NormalTok{alpha, beta))}
                \NormalTok{beta }\OperatorTok{=} \BuiltInTok{min}\NormalTok{(beta, best)}
                \NormalTok{root.beta }\OperatorTok{=} \NormalTok{beta}
                \ControlFlowTok{if} \NormalTok{beta }\OperatorTok{<=} \NormalTok{alpha:}
                    \NormalTok{pruned }\OperatorTok{=} \VariableTok{True}
        \BuiltInTok{print}\NormalTok{(root)}
        \ControlFlowTok{return} \NormalTok{best}
\end{Highlighting}
\end{Shaded}

    \section{Simulated Annealing}\label{simulated-annealing}

    \section{Genetic Algorithms}\label{genetic-algorithms}

    \begin{Verbatim}[commandchars=\\\{\}]
{\color{incolor}In [{\color{incolor} }]:} 
\end{Verbatim}


    % Add a bibliography block to the postdoc
    
    
    
    \end{document}
